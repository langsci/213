\addchap{Glossary}

New concepts or terminology introduced in this book. The first occurrence in each chapter is highlighted in small capitals.

\vspace{1cm}
\begin{footnotesize}
\begin{longtable}{p{4cm} p{8cm}}
\renewcommand{\arraystretch}{1.5}
\textsc{action-to-position}  & Two-stage multi-verb construction consisting of an action verb in V$_1$ and a positional verb in V$_fin$. See §\ref{sec:action-to-position} \\

\textsc{cause-result}  & Two-stage multi-verb construction consisting of a causing verb in V$_1$ and a resultant verb in V$_2$. See §\ref{sec:cause-result} \\

\textsc{clause-level event (cle)}  & Compositional event schema in which a set of lexical items projects one or more eventualities (e.g., two event stages) within what seems to be one clause. Consists of predicate-level events (PLE) and lexeme-level events (LLE). See §\ref{sec:event-typology} for discussion \\
 
\textsc{component-relating construction (crel)}  & Multi-verb construction type in which two or more verbs merge identical parts of their sublexical structure (for instance, motion semantics). Results in a single-stage MVC. See §\ref{sec:merging} for theoretical discussion, and §\ref{sec:crel} for examples and constructions \\

\textsc{direction complex}  & Single-stage multi-verb construction consisting of an action or perception verb in V$_1$ and a motion verb in V$_2$. See §\ref{sec:direction} \\
 
\textsc{event stage}  & A spatiotemporally definable eventuality with clearcut boundaries licensed by a verb's event argument. See discussion in §\ref{sec:davidsonian} \\
 
\textsc{discourse situation}  & Event construal on a level higher than the clause. See §\ref{sec:event-typology} for discussion \\
 
\textsc{free juxtaposition construction (fjux)}  & Multi-verb construction type in which two or more verbs interact in rather loose ways, i.e., without triggering obligatory argument interaction, operator harmonisation or restricting the use of conjunctions. Results in a two-stage MVC. See §\ref{sec:juxtaposition} for theoretical discussion, and §\ref{sec:fjux} for examples and constructions \\

\textsc{handling-to-action}  & Two-stage multi-verb construction consisting of a handling verb (mostly \textsc{take} verbs) in V$_1$ and an action verb in V$_fin$. See §\ref{sec:handling-to-action} \\

\textsc{handling-to-placement}  & Two-stage multi-verb construction consisting of a handling verb (mostly \textsc{take} verbs) in V$_1$ and a placement verb in V$_fin$. See §\ref{sec:handling-to-placement} \\
 
\textsc{juxtaposition} & Semantic technique of relating two or more verbs with each other. Neither staging of event schemas nor merging of sublexical features takes place. Results in free juxtaposition constructions. See §\ref{sec:juxtaposition} for examples and discussion \\
 
\textsc{lexeme-level event (lle)}  & Lexicon-driven event schema constituting the minimal eventuality in simplex predicates. Combines to form higher order event schemas in MVCs. See discussion in §\ref{sec:event-typology} \\
 
\textsc{merging}  & Semantic technique of matching sublexical features in verbs. Results in component-relating multi-verb constructions. See §\ref{sec:merging} for examples and discussion \\
 
\textsc{modification} & Semantic technique of combining the lexeme-level events of a matrix verb with a modifier verb (i.e., a verb with an unbound event argument). Results in modifying multi-verb constructions. See §\ref{sec:modification} for examples and discussion \\
 
\textsc {modifying construction}  & Multi-verb construction type in which the event argument of a matrix verb is copied to the event argument of a modifier verb which is assumed to be empty or unspecified at the lexicon level. Results in a single-stage MVC. See §\ref{sec:modification} for theoretical discussion, and §\ref{sec:modifying} for examples and constructions  \\

\textsc{motion complex}  & Single-stage multi-verb construction consisting of two or more motion verbs. See §\ref{sec:motioncomplex} \\

\textsc{motion-to-action}  & Two-stage multi-verb construction with a motion verb in V$_1$ and an active verb in V$_fin$. See §\ref{sec:motion-to-action} \\
 
\textsc{multi-verb construction (mvc)}  & Construction of two or more verboid elements that (i) predicate lexical content and select/assign arguments, (ii) lack constituent level differences or dependency hierarchies, (iii) are not connected by linking elements, (iv) form a coherent prosodic unit, and (v) entail one continuous time frame without disruptions. See §\ref{sec:defining} for further discussion  \\

\textsc{position-action}  & Two-stage multi-verb construction consisting of a posture verb in V$_1$ and an action verb in V$_fin$. Always triggers a co-temporal reading with both stages understood as occurring at the same time. See §\ref{sec:position-action} \\
 
\textsc{predicate-level event (ple)}  & Compositional event schema in which a set of lexical items together project one eventuality within what seems to be one predicate. Consists of lexeme-level events (LLE). See §\ref{sec:event-typology} \\

\textsc{sequitive complex}  & Single-stage multi-verb construction consisting of a verb of following or pursuit, and a motion verb. See §\ref{sec:sequitive} \\

\textsc{speech act complex}  & Single-stage multi-verb construction consisting of a speech act verb in V$_1$, and a \textsc{say} verb in V$_2$. See §\ref{sec:speechactcomplex} \\
 
\textsc{stacked MVCs}  & Hierarchically structured multi-verb construction hosting in one of its constructional slots another MVC on a lower level. See discussion in §\ref{sec:stackedmvcs} \\ 
 
\textsc{stage}  & See event stage \\
 
\textsc{stage-relating construction (srel)}  & Multi-verb construction type in which two or more verbs interact in rather tight ways, i.e., triggering obligatory argument interaction, operator harmonisation and constituent order. Results in a two-stage MVC with each verb projecting its own event stage. See §\ref{sec:staging} for theoretical discussion, and §\ref{sec:stage-relating} for examples and constructions  \\
 
\textsc{staging}  & Semantic technique of temporally adjoining two event stages licensed by event-denoting verbs. Results in stage-relating multi-verb constructions. See §\ref{sec:staging} for examples and discussion \\

\textsc{transport complex}  & Single-stage multi-verb construction consisting of a verb of transport (such as \textsc{bring} or \textsc{carry}) in V$_1$ and a motion verb in V$_2$. See §\ref{sec:transport} \\
\end{longtable} 
\end{footnotesize}