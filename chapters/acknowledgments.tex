\addchap{Acknowledgments}
\begin{refsection}

If tales grow in the telling\footnote{as J.R.R Tolkien put it in his introduction to the \textit{Lord of the Rings}}, then, certainly, dissertations grow in the writing. The writing of this thesis in particular took an awful amount of time, and would ultimately have failed if not for the kind and encouraging help from many teachers, colleagues, and friends. I am grateful to all of them for their support, and - not least - for their many ways, subtle and not so subtle, that would make me keep going on.

First and foremost, I would like to express my sincere thanks to my supervisors, Nikolaus P. Himmelmann, Birgit Hellwig, and Gerrit J. Dimmendaal (who became my third supervisor only at the very end but was very kind and supportive). Nikolaus has been with me for almost my entire linguistic life, supporting and encouraging me with unfailing confidence, and shaped much of what would later become my general linguistic background. I owe to him most of my comprehension of language and linguistics. It was Nikolaus who first stirred my interest in typology, prosody, serial verbs, and, not least, in Austronesian and Papuan languages. When, at the end of my undergraduate years, he offered me a PhD position in his DoBeS language documentation project on Wooi, I gladly accepted. Little did I know on which extraordinary journey I was to embark, a journey to a most fascinating country, and a most fascinating people. I am extremely thankful for Nikolaus' constant and generous support and care, as well as for the opportunity to be part of his fieldwork team in Indonesia. The experience will stay with me all my life.

Birgit came to Cologne some years after the start of my graduate study, but she soon proved, just like Nikolaus, to be as helpful and supportive a supervisor as one might wish for. I wish to thank Birgit for being an excellent "good cop", in particular for renewing my confidence when I needed it the most. Her specialist knowledge of various hard-to-grasp topics such as (typological) semantics, predicate decomposition and Papuan linguistics provided me with a most helpful vantage point for my own research and reasoning, and improved my understanding of multi-verb constructions in Eastern Indonesia in crucial ways. I'd like to thank Birgit also for many happy events outside office, enthusiastic discussions about certain death stars, and, of course, the jolly occasions at which she would serve her to-die-for Nigerian pepper soup to a bunch of hungry fellow linguists.

These fellow linguists, occasionally crowding into Birgit's or another one's appartment, were an essential and most joyful part of my graduate life in Cologne. Colleagues and fellow graduates soon turned into friends, making me feel at home from the very first day (so that working in the linguistics department in fact never quite felt like working at all). I wish to express my sincere thanks to all of them for creating and maintaining this wonderful and most stimulating working environment. In particular, I would like to thank Uta Reinöhl, Sonja Riesberg, Freya Morigerowsky, Meytal Sandler, Luh Anik Mayani, Ingeborg Fink, Felix Rau, Jan Strunk, Sonja Gipper, Katja Hannß, Henrike Frye, Carmen Dawuda, Isabel Compes, Gabriele Schwiertz, Aung Si, Kurt Malcher and Steffen Reetz for collective beer brewing (and drinking), long and most enjoyable bicycle tours, for gardening (and, of course, for après-garden parties), cooking (and dining), discussions on stingless bees and much more. Life within the confines of Meister-Ekkehart-Straße, E-Raum, Mensa and Zülpicher Straße was a most happy one.

During my field trips to Indonesia I enjoyed the hospitality and kindness of many people, specifically of our ``foster family" Ibu Min, Pak Jemy, Juen, Virgin, Rendi, Reni, their children and relatives. At CELD office, I fondly remember ``anak-anaknya" Eni Apriani, Sutri Narfarfan, Manu Tuturop, Novi Ndiken, Kris Walianggen, Hari Kristanto, Jimmi Karter Kirihio, Boas Wabia and Jean Lekeneny. Jimmi Karter Kirihio has been a most dedicated, mindful and knowledgeable language consultant, and no less a reliable friend. During my stay in Wooi village, I enjoyed the hospitality of Nehemia Werimon and his family. Freya Morigerowsky, Pak Romelus Wihyawari, the village chiefs (kepala desa) Abraham Werimon and Noak Wihyawari, among many other people, also proved to be most kind and helpful. Banyak terima kasih kepada kalian semua karena menerima saya dengan baik, membantu dengan soal-soal keseharian, menjaga kesehatan saya, serta mengajarkan saya bahasa Indonesia. Saya merasa bahagia karena pernah mengenal kalian dan sangat menghargai waktu kebersamaan itu. Enak sampe.

Special thanks are due to my proofreaders who turned a manuscript into what finally felt like a ``real thesis": Sonja Riesberg, Rebecca Defina, Uta Reinöhl, Luh Anik Mayani, and, in particular, Aung Si who did a tremendously painstaking job by correcting and amending all my shortcomings in grammar and style. Rebecca Defina, Antoinette Schapper, Felix Ameka, Wayan Arka, Uta Reinöhl, Luh Anik Mayani, Felix Rau, Sonja Riesberg and Kurt Malcher at various points discussed topics of my thesis with me, all of which greatly improved its quality. Henrike Frye, Ingeborg Fink and Jan Strunk were so kind as to help me get my mind around LaTeX and R. Many thanks are due to two anonymous reviewers from Language Science Press for many helpful comments and suggestions, as well as to the editor of the SDiL series, Martin Haspelmath, for very helpful advice and for accepting a rather voluminous and not-so-easy-to-digest manuscript into the series.

Last but certainly not least I would like to thank my friends and family outside linguistics (if there really is such a thing). Stefan Bauhaus, Gerda Moritz, Anna Hanelt, Catha Capteyn, René Grotegerd and Sarah Verlage (née Dolscheid) have been with me all the way since my undergraduate days in Münster. Birgit Röttering, Michael Lakermann, Hubert Sumser, Armin Jagel and Corinne Buch accompanied me when, in later years, my mind would wander astray into the realms of botany, entomology and meadow conservation. Finally, I am immensely grateful for the unlimited love and support from my parents, Hanna and Wolfgang Unterladstetter.

\printbibliography[heading=subbibliography]
\end{refsection}