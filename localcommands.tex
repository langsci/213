\newcommand{\acs}[1]{{\color{red}#1}}

\newcommand{\glft}{\glt}

\renewcommand{\a}{\ea}
% 
\newcommand{\pex}{}
\newcommand{\xe}{}
%\AtBeginDocument{\renewcommand{\gla}{\gll}}
\newcommand{\glb}{\color{green}}
\newcommand{\glc}{\color{red}}
\newcommand{\gld}{\color{orange}}
\newcommand{\trailingcitation}{\hfill}
\newcommand{\nogloss}[1]{{\color{pink}#1}}

% \def\begingl{\relax}
% \renewcommand{\begingl}{\ea}
% \def\endgl{\relax}
% \newcommand{\endgl}{\z}
% \renewcommand{\endgl}{\z}

%epigraph settings
\makeatletter
\newenvironment{chapquote}[2][2em]
  {\setlength{\@tempdima}{#1}%
   \def\chapquote@author{#2}%
   \parshape 1 \@tempdima \dimexpr\textwidth-2\@tempdima\relax%
   \itshape}
  {\par\small\hfill---\ \chapquote@author\hspace*{\@tempdima}\par\bigskip}
\makeatother


%% Glossary entries
\setglossarypreamble{New concepts or terminology introduced in this book. The first occurrence in each chapter is highlighted in small capitals.\bigskip\par}
\newglossaryentry{Action-to-position}{name={Action-to-position},description={Two-stage multi-verb construction consisting of an action verb in V$_1$ and a positional verb in V$_{fin}$. See §\ref{sec:action-to-position}.}}
\newglossaryentry{Cause-result}{name={Cause-result},description={Two-stage multi-verb construction consisting of a causing verb in V$_1$ and a resultant verb in V$_2$. See §\ref{sec:cause-result}.}}
\newglossaryentry{Clause-level event (CLE)}{name={Clause-level event (\textsc{CLE})}, description={Compositional event schema in which a set of lexical items project one or more eventualities (e.g., two event stages) within what seems to be one clause. Consists of predicate-level events (PLE) and lexeme-level events (LLE). See §\ref{sec:event-typology} for discussion.}}
\newglossaryentry{Component-relating construction (CREL)}{ name={Component-relating construction (\textsc{CREL})}, description={Multi-verb construction type in which two or more verbs merge identical parts of their sublexical structure (for instance, motion semantics). Results in a single-stage MVC. See §\ref{sec:merging} for theoretical discussion, and §\ref{sec:crel} for examples and constructions.}}
\newglossaryentry{Direction complex}{ name={Direction complex}, description={Single-stage multi-verb construction consisting of an action or perception verb in V$_1$ and a motion verb in V$_2$. See §\ref{sec:direction}.}}
\newglossaryentry{Discourse situation}{ name={Discourse situation}, description={Event construal at a level higher than the clause. See §\ref{sec:event-typology} for discussion.}}
\newglossaryentry{Event stage}{ name={Event stage}, description={A spatiotemporally definable eventuality with clearcut boundaries licensed by a verb's event argument. See discussion in §\ref{sec:davidsonian}.}}
\newglossaryentry{Free juxtaposition construction (FJUX)}{ name={Free juxtaposition construction (\textsc{FJUX})}, description={Multi-verb construction type in which two or more verbs interact in rather loose ways, i.e., without triggering obligatory argument interaction, operator harmonisation or restricting the use of conjunctions. Results in a two-stage MVC. See §\ref{sec:juxtaposition} for theoretical discussion, and §\ref{sec:fjux} for examples and constructions.}}
\newglossaryentry{Handling-to-action}{ name={Handling-to-action}, description={Two-stage multi-verb construction consisting of a handling verb (mostly \textsc{take} verbs) in V$_1$ and an action verb in V$_{fin}$. See §\ref{sec:handling-to-action}.}}
\newglossaryentry{Handling-to-placement}{ name={Handling-to-placement}, description={Two-stage multi-verb construction consisting of a handling verb (mostly \textsc{take} verbs) in V$_1$ and a placement verb in V$_{fin}$. See §\ref{sec:handling-to-placement}.}}
\newglossaryentry{Juxtaposition}{name={Juxtaposition}, description={Semantic technique of relating two or more verbs with each other. Neither staging of event schemas nor merging of sublexical features takes place. Results in free juxtaposition constructions. See §\ref{sec:juxtaposition} for examples and discussion.}}
\newglossaryentry{Lexeme-level event (LLE)}{ name={Lexeme-level event (\textsc{LLE})}, description={Lexicon-driven event schema constituting the minimal eventuality in simplex predicates. Combines to form higher order event schemas in MVCs. See discussion in §\ref{sec:event-typology}.}}
\newglossaryentry{Merging}{ name={Merging}, description={Semantic technique of matching sublexical features in verbs. Results in component-relating multi-verb constructions. See §\ref{sec:merging} for examples and discussion.}}
\newglossaryentry{Modification}{name={Modification}, description={Semantic technique of combining the lexeme-level events of a matrix verb with a modifier verb (i.e., a verb with an unbound event argument). Results in modifying multi-verb constructions. See §\ref{sec:modification} for examples and discussion.}}
\newglossaryentry{Modifying construction}{ name={Modifying construction}, description={Multi-verb construction type in which the event argument of a matrix verb is copied to the event argument of a modifier verb which is assumed to be empty or unspecified at the lexicon level. Results in a single-stage MVC. See §\ref{sec:modification} for theoretical discussion, and §\ref{sec:modifying} for examples and constructions.}}
\newglossaryentry{Motion complex}{ name={Motion complex}, description={Single-stage multi-verb construction consisting of two or more motion verbs. See §\ref{sec:motioncomplex}.}}
\newglossaryentry{Motion-to-action}{ name={Motion-to-action}, description={Two-stage multi-verb construction with a motion verb in V$_1$ and an active verb in V$_{fin}$. See §\ref{sec:motion-to-action}.}}
\newglossaryentry{Multi-verb construction (MVC)}{ name={Multi-verb construction (\textsc{MVC})}, description={Construction of two or more verboid elements that (i) predicate lexical content and select/assign arguments, (ii) lack constituent level differences or dependency hierarchies, (iii) are not connected by linking elements, (iv) form a coherent prosodic unit, and (v) entail one continuous time frame without disruptions. See §\ref{sec:defining} for further discussion .}}
\newglossaryentry{Position-action}{ name={Position-action}, description={Two-stage multi-verb construction consisting of a posture verb in V$_1$ and an action verb in V$_{fin}$. Always triggers a co-temporal reading with both stages understood as occurring at the same time. See §\ref{sec:position-action}.}}
\newglossaryentry{Predicate-level event (PLE)}{ name={Predicate-level event (\textsc{PLE})}, description={Compositional event schema in which a set of lexical items together project one eventuality within what seems to be one predicate. Consists of lexeme-level events (LLE). See §\ref{sec:event-typology}.}}
\newglossaryentry{Sequitive complex}{ name={Sequitive complex}, description={Single-stage multi-verb construction consisting of a verb of following or pursuit, and a motion verb. See §\ref{sec:sequitive}.}}
\newglossaryentry{Speech act complex}{ name={Speech act complex}, description={Single-stage multi-verb construction consisting of a speech act verb in V$_1$, and a \textsc{say} verb in V$_2$. See §\ref{sec:speechactcomplex}.}}
\newglossaryentry{Stacked MVCs}{ name={Stacked MVCs}, description={Hierarchically structured multi-verb construction hosting in one of its constructional slots another MVC at a lower level. See discussion in §\ref{sec:stackedmvcs}.}} 
\newglossaryentry{Stage}{ name={Stage}, description={See \textit{event stage}.}}
\newglossaryentry{Stage-relating construction (SREL)}{ name={Stage-relating construction (\textsc{SREL})}, description={Multi-verb construction type in which two or more verbs interact in rather tight ways, i.e., triggering obligatory argument interaction, operator harmonisation and constituent order. Results in a two-stage MVC, with each verb projecting its own event stage. See §\ref{sec:staging} for theoretical discussion, and §\ref{sec:stage-relating} for examples and constructions .}}
\newglossaryentry{Staging}{name={Staging}, description={Semantic technique of temporally adjoining two event stages licensed by event-denoting verbs. Results in stage-relating multi-verb constructions. See §\ref{sec:staging} for examples and discussion.}}
\newglossaryentry{Transport complex}{name={Transport complex}, description={Single-stage multi-verb construction consisting of a verb of transport (such as \textsc{bring} or \textsc{carry}) in V$_1$ and a motion verb in V$_2$. See §\ref{sec:transport}.}}


\newcommand{\ilit}[1]{#1\il{#1}}


\renewcommand{\lsImpressumExtra}{This book is the revised version of the author's doctoral dissertation, which was completed in
2018 at the University of Cologne.}
